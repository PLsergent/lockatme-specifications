\part{Software Design Specifications}
\section{Utilisation}
  \subsection{Installation Qualification}
Pour installer le logiciel lockatme, les utilisateurs auront deux options. Le
projet sera certainement upload sur pip. Il pourra donc utiliser la commande
suivante disponible sur toutes les distributions Linux, à condition d'avoir
installé le package pip de python :
\begin{lstlisting}[language=bash]
  ~$ pip install lockatme
\end{lstlisting}
Sinon il pourra tout simplement cloner le projet depuis GitHub en réalisant la
commande suivante :
\begin{lstlisting}[language=bash]
  ~$ git clone https://github.com/lockatme/lockatme-specifications.git
\end{lstlisting}
Le dossier comprendra un fichier exécutable en Python que l'utilisateur pourra
lancer avec la ligne suivante dans son shell :
\begin{lstlisting}[language=bash]
  ~$ python lockatme
\end{lstlisting}

  \subsection{Explication générale}
Une fois l'installation réalisée, comme expliqué dans la partie précédente,
le fichier exécutable \emph{lockatme.py} sera présent dans le dossier du logiciel. Son
exécution aura comme effet de verrouiller l'écran. Par défaut l'écran sera
recouvert d'un filtre transparent avec un cadenas (voir figure 4). A noter que
durant cette période de verrouillage, l'écran de l'ordinateur pourra très bien
se mettre en veille (écran noir) selon les configurations initiales de
l'utilisateur.
Pour sortir du verrouillage l'utilisateur aura juste à appuyer sur une touche
du clavier. Ensuite la reconnaissance faciale commencera immédiatement (voir
figure 5).
Un point rouge apparaissant et disparaissant indiquera une prise de
photo de la part de la webcam. Si une photo prise ainsi correspond à une photo
modèle, l'écran se déverrouillera. Au bout d'un moment (configurable dans le
fichier \emph{config.py}), si le déverrouillage échoue l'utilisateur sera invité à
entrer son mot de passe (voir figure 6).

  \subsection{Exemple sous i3}
    \subsubsection{Photos modèles}
Le choix des photos modèles devra se faire de manière autonome. À aucun moment
le logiciel invitera l'utilisateur à mettre des photos modèles dans le dossier
prévu cet effet. De même, pour le mot de passe devra être défini dans le
fichier \emph{password.py}. En cas d'absence de photos
modèles le logiciel demandera un mot de passe directement. Si le mot de passe
n'a pas été défini il suffira d'appuyer sur Entrée.
Voici la marche à suivre :
\begin{lstlisting}[language=bash]
  ~$ cp path/to/image ~/lockatme/images/
\end{lstlisting}
À noter qu'il est possible de mettre plusieurs photos modèles, dans la mesure
ou si l'image de la webcam correspond à l'une des photos le déverrouillage
fonctionnera.
\vspace{0.5cm}
Bien évidemment une image de bonne qualité est préférable, le logiciel ne
vérifiera pas la qualité de l'image. L'utilisateur devra donc voir à l'usage.

    \subsubsection{Raccourci clavier}
Cet exemple est réalisé à partir d'une distribution ArchLinux avec le Windows
manager i3. La marche à suivre est donc susceptible de changer.
\begin{lstlisting}[language=bash]
  ~$ cd .config/i3/
  ~/.config/i3/$ atom config
\end{lstlisting}
Exemple de config :
\begin{tcolorbox}[enhanced,width=5in,center upper,
    fontupper=\large\bfseries,drop fuzzy shadow southwest,
    boxrule=0.4pt,sharp corners,colframe=yellow!80!black,colback=yellow!10]
    program bindkey\\
    bindsym mod+o exec opera\\
    bindsym mod+g exec chromium\\
    bindsym mod+i exec spotify\\
    bindsym mod+p exec libreoffice\\
    bindsym mod+Shift+f exec firefox\\
    \emph{bindsym mod+Key+binding exec lockatme.py}
\end{tcolorbox}
    \subsubsection{Fermeture clapet}

\section{Diagramme}
\section{Bibliothèques utilisées}
\section{Exemple code}
