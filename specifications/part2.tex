\chapter{Présentation du Projet}
\newpage

\section{Buts}
Le but premier de l’application est de déverrouiller un écran d’ordinateur, à
l’aide d’une caméra, par reconnaissance faciale. Cependant cela implique de
mettre en place un verrouillage d’écran sous Linux. Les URS spécifiques seront
décrit plus tard dans ce document.

\section{Motivations}
Trois membres du groupe utilisent Arch Linux qui est une distribution minimale
de Linux. Le fait de quitter Windows leur a permis de pleinement se concentrer
sur la machine à un plus bas niveau, avec tous les avantages de liberté
qu’offre une plateforme open source, mais aussi toutes les contraintes qui sont
très formatrices et qui forcent à se pencher d’avantage sur le fonctionnenement
de ce système d’exploitation. Les trois utilisateurs cherchaient une manière de
verrouiller/déverrouiller leur écran de manière sécurisée. Et l’idée de ce
projet a fleuri suite à un article présent dans le magazine Linux
Magazine/France n{\degree}203 : “Mettez en place un système de reconnaissance faciale”.

\section{Linux}
Le développement du logiciel se fera sur Linux. Un tel projet
sur Windows aurait été bien plus difficile concernant l’implémentation système
mais aussi le code de l’application. De plus, l’OS est largement privilégié par
les développeurs dans le monde de la programmation. C’est pourquoi nous avons
choisi de réaliser notre projet sous Linux, qui s’adressera donc à un public
famillié avec la CLI (Command Line Interface) et les autres aspects techniques.
Des interfaces seront potentiellement développées à terme pour les utilisateurs
de distributions plus user-friendly (comme Ubuntu).

\section{Open Source}
Le développement du projet se fera de manière complètement transparente et donc
en open source. Ce choix est assez logique lorsque l’on réalise un programme
pour Linux, car il s’inscrit exactement dans la politique des développeurs qui
ont réalisé ce dernier. Cela possède de nombreux avantages : possibilité pour
la communauté de contribuer au projet au travers de modifications du code,
commentaires, rapport de bug, \ldots
