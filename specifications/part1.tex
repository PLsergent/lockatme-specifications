\chapter{Présentation des membres du groupe}
\newpage

\section{Contexte}
Dans le cadre de notre DUT Informatique à l’IUT Lyon 1, nous sommes tenus de
réaliser un projet tuteuré durant le second semestre. Ce projet s’étendant
également sur le troisième semestre, il a pour but de répondre à une
problématique précise et de mettre en oeuvre les compétences acquises au cours
de la formation. Il a aussi vocation à faire découvrir de nouveaux domaines et
il nous permettra d’élargir nos savoirs à travers une auto-formation.

\vspace{0.5cm}
Ce projet se découpe en deux axes :
\begin{itemize}
  \item{Rédaction du cahier des charges (second semestre)}
  \item{Réalisation du projet en lui même (troisième semestre)}
\end{itemize}
\vspace{0.5cm}

Malgré une liste de sujets proposés, notre groupe a voulu suivre ses propres
motivations (présentées plus loin dans ce document) et a choisi de proposer un
sujet à M. Vidal. L’intitulé est le suivant : Verrouillage et déverrouillage
d’écran par reconnaissance faciale sous Linux.

\section{Organisation et membres}
L’équipe chargée de ce projet est constituée
\begin{itemize}
  \item{Tuteur du projet : M. Vincent VIDAL}
  \item{Chef du projet : M. Bruno INEC}
  \item{Membres : M. David ANANDANADARADJA, Mme Sagar GUEYE, M. Matthieu
        KIRSCHLEGER et M. Pierre-Louis SERGENT}
\end{itemize}

\section{Compétences}
Notre projet comporte deux contraintes principales : il nécessite une bonne
connaissance du langage Python (explication choix outils cf III) et une
maîtrise de Linux. L’impulsion de ces choix vient en grande partie du chef de
projet qui possède une expérience importante dans ces deux domaines. David et
Pierre-Louis possèdent quant à eux une expérience modérée dans l’utilisation de
Linux (distribution Arch). L’ensemble des compétences individuelles est résumé
ci-après :

\newpage

\textbf{Python} :
\begin{itemize}
  \item{Confirmé : Bruno INEC}
  \item{Intermédiaire : Pierre-Louis SERGENT}
  \item{Débutant : Sagar GUEYE, Matthieu KIRSCHLEGER, David ANANDA}
\end{itemize}

\textbf{Linux} :
\begin{itemize}
  \item{Confirmé : Bruno INEC}
  \item{Intermédiaire : David ANANDA, Pierre-Louis SERGENT}
  \item{Débutant : Sagar GUEYE, Matthieu KIRSCHLEGER}
\end{itemize}

Comme le montre le listing précédent, les compétences du groupe sont très
disparates. Cela peut apparaître comme une contrainte, mais en réalité cela
constitue une véritable opportunité pour tous les membres. Ils vont ainsi
pouvoir se former dans les domaines ci-après. Ils sont essentiels pour la
suite des études et pour le milieu professionnel.

\vspace{0.5cm}
\begin{itemize}
  \item{Programmation : Linux, Python}
  \item{Rédaction cahier des charges, \LaTeX}
  \item{Travail en équipe : réunion, communication, CI, modèle de
  développement}
\end{itemize}
\vspace{0.5cm}

Nous étions donc motivés pour nous lancer dans un sujet avec nombre d’inconnus
mais qui allait être fort enrichissant.
\vspace{0.3cm}

À noter également que dans un projet s'étendant sur une telle durée les
compétences humaines et plus généralement les compétences annexes à
l’informatique ne sont pas à négliger.
