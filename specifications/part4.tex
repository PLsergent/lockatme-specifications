\part{Phase du projet -- Distribution des tâches}
\section{Contexte}
Le projet repose, en assez grande partie, sur des domaines qu’ils nous
fallaient découvrir et étudier. C’est pourquoi l’attribution et la définition
des tâches à été une problématique importante dans le projet lockatme.
Cependant, je profite de cette introduction pour rappeler que cette étape est
essentielle pour le bon déroulement du développement. Nous nous sommes efforcés
d’attribuer les tâches à des duos ou trios afin de permettre deux choses : une
auto-formation étant parfois nécessaire, les membres assignés à une même tâche
pouvaient partager les connaissances et les difficultés, afin de faire avancer
le développement plus vite. De plus les membres ne devraient pas rester bloquer
sur une tâche étant donné que les attributions se font en fonction des
compétences de chacuns.

\section{Listes des tâches}
La liste des tâches est fortement susceptible d’évoluer au cours du projet en
fonction de l’évolution de celui-ci, des difficultés recontrées et du temps
imparti. La réalisation des tests n’est pas spécifiée dans cette liste.
Niveaux d’importances: 1 – essentiel, 2 - important, 3 – importance modérée,
4 – optionnel.

T1 : Cahier des charges(1)
\begin{itemize}
  \item{conception}
  \item{rédaction}
  \item{validation}
\end{itemize}

T2 : Déverrouillage
\begin{itemize}
  \item{algo reconnaissance faciale(1)}
  \item{algo déverrouillage par mot de passe(1)}
  \item{gestion des cas d’erreurs(1)}
  \item{fichier configuration/personnalisation(2)}
\end{itemize}

T3 : Implementation système, installation
\begin{itemize}
  \item{verrouillage écran(1)}
  \item{déverrouillage écran(1)}
  \item{installation qualification(2)}
  \item{upload packages sur pip/yaourt(3)}
  \item{readme(2)}
\end{itemize}

T4 : Amélioration
\begin{itemize}
  \item{interface graphique(3)}
  \item{développemement plateforme commune de verrouillage/déverrouillage
        d’écran(4)}
\end{itemize}
