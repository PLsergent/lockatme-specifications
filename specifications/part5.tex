\chapter{User Requirement Specifications}
\newpage

\section{Définition}
Les \emph{User Requirement Specifications} (URS), ou le cahier des charges des
spécifications de l’utilisateur, a pour but de présenter les exigences du
client ou de l’utilisateur. Il doit contenir une liste exhaustive
(dans la mesure du possible) de ce que le logiciel sera en mesure de faire au
terme de son développement. Dans notre cas, les exigences sont tout d’abord
posées par nous, premiers utilisateurs. À terme, nous espérons élargir cela à
une communauté se formant autour du programme.

\section{Priorité}
Nous avons décidé de diviser les URS en trois niveaux de priorité afin de bien
séparer les caractéristiques que nous voulons absolument trouver dans le
logiciel final et les caractéristiques qui pourraient être intéressantes à
développer dans le futur.
\begin{itemize}
  \item{M – Mandatory requirement. **doit être développé**
‘Cette caractéristique doit être incluse dans le logiciel final.’}
  \item{D – Desirable requirement. **doit être développé dans la mesure du
  possible**
‘Cette caractéristique devrait être incluse dans le logiciel final, sauf si
contrainte trop importante ou manque de temps.’}
  \item{E – Possible future enhancement. **peut être développé dans le futur
  ou sera développé en fonction de l’avancement du projet**
‘Cette caractéristique pourrait être incluse dans le logiciel final dans le
futur.’}
\end{itemize}

\section{Mandatory requirement}
  \subsection{Explication générale (1)}
Le logiciel lockatme devra être installable sur Linux et aura pour
fonctionnalité de permettre à l’utilisateur de vérrouiller son écran puis de
le déverrouiller par reconnaissance faciale ou, le cas échéant, par mot de
passe. Pour se faire l’utilisateur sera invité à enregistrer des photos de
lui dans un dossier spécifique, ces photos seront utilisées comme modèle pour la
reconnaissance par webcam. Des tests devront être effectués pour déterminer si ces
photos sont de qualités suffisantes pour reconnaître l’utilisateur à coup sûr.
Le paramétrage du programme se fera à travers un fichier qui correspond aux
normes actuelles des fichiers de configuration.

  \subsection{Organisation (1)}
  \begin{itemize}
    \item{M.1 : Verrouillage écran}
    L’utilisateur sera capable de verrouiller son écran avec une commande.
    \\
    \item{M.2 : Reconnaissance faciale}

    Le logiciel sera capable de reconnaître le visage de l’utilisateur à
    l’aide d’une webcam.
    \\
    \item{M.3 : Déverrouillage écran}

    L’utilisateur sera capable de déverrouiller son écran.
    \begin{itemize}
      \item{M.3.1 Par reconnaissance faciale}
      \begin{itemize}
        \item{M.3.1.1 Prise d’information}

        Le logiciel sera capable de prendre une photo toutes les trois
        secondes à travers la webcam.
		    Le cas échéant cf M.3.2
        \end{itemize}
        \vspace{0.5cm}
      \item{M.3.2 Par saisie de mot de passe}

      En cas d’échec de la reconnaissance faciale, l'utilisateur sera rédirigé
      vers un déverrouillage d'écran par mot de passe.
      \end{itemize}
  \end{itemize}

\section{Desirable requirement}
  \subsection{Explication générale (2)}
  De plus nous souhaiterions que l’utilisateur puisse personnaliser plusieurs
  aspects du logiciel. Par exemple, il lui serait possible de choisir une image
  lors du verrouillage de l’écran. Il pourrait également modifier la fréquence
  de prise d’information lors du déverrouillage par reconnaissance faciale.
  Concernant le verrouillage de l’écran, nous souhaiterions qu’il se fasse de
  manière automatique lorsque l’utilisateur baisse le clapet, ou qu’il appuie
  sur une touche choisie.
  Nous souhaiterions aussi réaliser une interface graphique
  pour le choix des images «modèles» et pour l’entrée du mot de passe.

  \subsection{Organisation (2)}
  \begin{itemize}
  \item{M.1 Verrouillage écran}
    \begin{itemize}
    \item{D.1 Verrouillage simplifié de l’écran}

    L’utilisateur devrait être capable de verrouiller son écran avec :
      \begin{itemize}
      \item{D.1.1 Un raccourci clavier *}
      \item{D.1.2 La fermeture clapet *}
      \\
      \end{itemize}
      \item{D.2 Personnalisation de l’écran verrouillé}

      L’utilisateur devrait être capable de choisir l’aspect 	de son écran
      verrouillé (image, texte).
      \\
    \end{itemize}
    \item{M.3.1.1 Prise d’information}
    \begin{itemize}
      \item{D.3 Choix fréquence prise photo}

      L’utilisateur devrait être capable de choisir la 	fréquence de prise de
      photos lors du déverrouillage par reconnaissance faciale.
      \\
    \end{itemize}
    \item{D.4 Interface graphique}

    L’utilisateur devrait être capable de choisir ses photos et entrer son mot
    de passe lors du déverrouillage grâce à une interface graphique.
  \end{itemize}
\vspace{0.5cm}

* : Il est important de rappeler que le logiciel sera disponible uniquement sur
Linux. L'utilisateur sera donc très libre pour les configurations, et l'utilisation
d'un raccourci clavier ou du verrouillage automatique par fermeture du clapet
ne pourra se faire qu'à travers des changements réalisés par l'utilisateur sur sa machine. La
marche à suivre sera indiquée dans le mode d'emploi. Mais l'utilisateur restera
très libre de ses actions.

\section{Possible future enhancement}
  \subsection{Explication générale (3)}
  Le projet étant open source nous invitons les développeurs à consulter le
  repository GitHub pour contribuer au projet et proposer de nouvelles
  fonctionnalités.
  L’idée du projet est que lockatme pourra être utilisé comme programme de
  base pour développer de nouveaux moyens de déverrouillage d’écran. Nous
  avons notamment pensé à un déverrouillage par finger-print (pour les
  ordinateurs équipés). Ainsi l’utilisateur aurait la possibilité de choisir
  parmis les différents moyen de déverrouillage pour sécuriser son ordinateur.
  Une autre avancée souhaitable serait la récupération de photos des réseaux
  sociaux (via leurs API respectives) afin que l’utilisateur n’ait pas à
  choisir des photos manuellement.
  Enfin pour garantir la sécurité du déverrouillage, il serait souhaitable
  d’améliorer la reconnaissance faciale, pour faire face à de possibles
  failles (montrer une photo de l’utilisateur à la webcam).
  \subsection{Organisation (3)}
  \begin{itemize}
  \item{M.3 Déverrouillage écran}
    \begin{itemize}
    \item{E.1 Programme modulaire}
      \begin{itemize}
      \item{E.1.1 Par Finger-print}

      L’utilisateur pourrait être capable déverrouiller son écran avec un
      finger-print.
      \\
      \end{itemize}
    \end{itemize}
    \item{M.2 Reconnaissance faciale}
    \begin{itemize}
    \item{E.2 Récupération de photos automatique}

    L’utilisateur pourrait être capable de passer par un réseau social
    pour récupérer ses photos et metadatas grâce à l'API.
    \\
    \end{itemize}
    \item{M.3.1 Déverrouillage écran par reconnaissance faciale}
    \begin{itemize}
      \item{E.3 Instruction spécifique}

      Le programme pourrait être capable de demander à l’utilisateur de faire
      certaines gestuelles spécifiques du visage.
    \end{itemize}
  \end{itemize}
