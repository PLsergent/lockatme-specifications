\part{Implémentation système}
\section{Interface standard}

Linux offre une interface standard pour l'authentification à travers la
bibliothèque PAM (Pluggable Authentication Module). Des modules peuvent être
écrit et utilisé par n'importe quel programme utilisant la bibliothèque.
Cela offre le grand avantage d'être très portable et modulaire. De plus, macOS
et Android pourraient ainsi être supporté.
En revanche, écrire un module PAM n'est pas des plus aisé. C'est un travail
assez bas niveau qui demande une bonne connaissance du ANSI-C.
Sans l'existance de bindings Python pour la bibliothèque, cette solution semble
être compliqué à réaliser pour notre groupe.

\section{Locker tier}

Une autre solution est d'utiliser un screenlocker qui permait de lock et unlock
au travers de commandes. Un exemple standard est l'application \emph{XScreenSaver}
qui permait de lock avec la commande 'xscreensaver-command -l' et d'unlock avec
'xscreensaver-command -d'.
Cela offre l'avantage d'offrir d'office la possibilité d'unlock avec un mot de
passe. De plus, nous pourrions donc utiliser toutes les possiblités qu'offrent
le programme.
Cette solution est moins portable et ne laisse pas autant de liberté, mais elle
est plus simple et plus rapide à implémenter.
